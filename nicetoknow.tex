%%%%%%%%%%%%%%%%%%%%%%%%%
% Dokumentinformationen %
%%%%%%%%%%%%%%%%%%%%%%%%%
% !TeX program = pdflatex
% !TeX encoding = utf8
% !TeX spellcheck = de_DE
\newcommand{\titleinfo}{Nice to Know}
\newcommand{\authorname}{\href{mailto:mgisler@hsr.ch}{M. Gisler} }
\newcommand{\authoremail}{\href{mailto:mgisler@hsr.ch}{M. Gisler} }

%BuG-Fix
%Package pdf Error: Driver file ................ not found
%If you have a luatex driver fail uncomment these lines
\RequirePackage{luatex85}
\def\pgfsysdriver{pgfsys-pdftex.def}
% Genereller Header
\documentclass[11pt,twoside,a4paper,fleqn]{article}
% Dateiencoding
%\usepackage{fontspec}
%\setmainfont{Calibri}
\usepackage[utf8]{inputenc}
\usepackage[T1]{fontenc}	%ä,ü...
% Seitenränder
\usepackage[left=1cm,right=1cm,top=0.5cm,bottom=0.5cm,includeheadfoot]{geometry}
% Sprachpaket
\usepackage[ngerman]{babel} % Silbentrennung und Rechtschreibung Englisch und Deutsch

%%%%%%%%%%%%%%%%%%%%%%%
%% Wichtige Packages %%
%%%%%%%%%%%%%%%%%%%%%%%
\usepackage{amsmath}                % Allgemeine Matheumgebungen									
\usepackage{amssymb}                % Fonts: msam,msbm, eufm & Mathesymbole, Mengen (lädt automatisch amsfonts)									
\usepackage{array}                  % \newcolumntype, \firsthline, ,\lasthline, m{width}, b{width}									
\usepackage{caption}                % Bildunterschriften									
\usepackage{enumitem}               % basic environments: enumerate, itemize, description									
\usepackage{fancybox}               % \fbox: \shad­ow­box, \dou­ble­box, \oval­box, \Oval­box									
\usepackage{fancyhdr}               % Seiten schöner gestalten, insbesondere Kopf- und Fußzeile									
\usepackage{floatflt}               % Textumflossene Abbildungen \begin{floatingfigure}[r]{Breite} : r rechts, l links, p links auf geraden Seiten und rechts auf ungeraden Seiten									
\usepackage{graphicx}               % \includegraphics[keyvals]{imagefile}, [draft]graphicx zeigt nur Namen und Rahmen an, [final] hebt diese option auf => Bild wird angezeigt    									
\usepackage{hyperref}               % Erstellt Verweise innerhalb und nach außerhalb eines PDF Dokumentes.									
\usepackage{lastpage}               % Bspw. : Page 1 of 3 => \thepage\ of \pageref{LastPage}									
\usepackage{listings}               % Erlaubt es Programmcode in der gewünschten Sprache zu hinterlegen (C++, Matlab,..). Definition der Sprache mit \lstset{language=name}..									
\usepackage{longtable}              % Longtable erlaubt es Tabellen zu erstellen die bei der nächsten Seite weiterlaufen. (Bricht automatisch um)									
\usepackage{mathabx}                % Mathesymbole									
\usepackage{mathrsfs}               % \mathscr (Benötigt für Fourierreihen-Symbol)									
%\usepackage{mathtools}              % Extension package to amsmath									
\usepackage{multicol}               % multicols-Umgebung \begin{multicols}{3} erzeugt Abschnitt mit 3 Spalten									
\usepackage{multirow}               % Tabelle: ermöglicht es Felder mehrerer Zeilen in einem zusammenzufassen									
\usepackage{pdflscape}              % adds PDF support to the environment 'landscape'									
\usepackage{pxfonts}                % Symbole, griechisches Alphabet, Integrale...									
\usepackage{rotating}               % sideways, turn{degree}, rotate{degree}, sidewaysfigure, sidewaystable Umgebung									
\usepackage{subcaption}             % Bildunterschriften für Subfigures									
\usepackage{tabularx}               % tabularx-Umgebung: Hat feste Gesamtbreite, \begin{tabularx}{\textwidth}{c c c c c} X: Spalte mit variabler Breite, l, c, r, p{breite}, m{breite}									
\usepackage{textcomp}               % text symbols: baht, bullet, copyright, musical-note, onequarter, section, yen									
\usepackage{tikz}                   % Tikz Umgebung zur Grafikerzeugung									
\usepackage{titlesec}               % Überschriften zu Textabstände
\usepackage{trfsigns}               % Transformationszeichen \laplace, \Laplace..									
\usepackage{trsym}                  % Weitere Laplace Zeichen erlaubt auch vertikale Transformationszeichen									
\usepackage{verbatim}               % verbatim, verbatim*, comment Umgebung									
\usepackage{wrapfig}                % Textumflossene Bilder und Tabellen, \begin{wrapfigure}[Zeilen]{Position}[Ueberhang]{Breite}									
\usepackage{xcolor}                 % \pagecolor{color}, \textcolor{color}{text}, \colorbox{color}{text}, \fcolorbox{border-color}{fill-color}{text}									
\usepackage{titlesec}
% Zum Bilder einfach in Tabellen einfügen (valign=t)
\usepackage[none]{hyphenat}
\usepackage[export]{adjustbox}
\usepackage{listings}
\usepackage{esint}
\usepackage{graphics}
\usepackage{pdfpages}
\usepackage{float}

%%%%%%%%%%%%%%%%%%%%
% Generelle Makros %
%%%%%%%%%%%%%%%%%%%%


\newcommand\tabbild[2][]{%
	\raisebox{0pt}[\dimexpr\totalheight+\dp\strutbox\relax][\dp\strutbox]{%
		\includegraphics[#1]{#2}%
	}%
}


%%%%%%%%%%
% Farben %
%%%%%%%%%%
\definecolor{black}{rgb}{0,0,0}
\definecolor{red}{rgb}{1,0,0}
\definecolor{white}{rgb}{1,1,1}
\definecolor{grey}{rgb}{0.8,0.8,0.8}
\definecolor{green}{rgb}{0,.8,0.05}
\definecolor{brown}{rgb}{0.603,0,0}
\definecolor{mymauve}{rgb}{0.58,0,0.82}


%%%%%%%%%%%%%%%%%%%%%%%%%%%%
% Mathematische Operatoren %
%%%%%%%%%%%%%%%%%%%%%%%%%%%%
\DeclareMathOperator{\sinc}{sinc}
\DeclareMathOperator{\sgn}{sgn}
\DeclareMathOperator{\Real}{Re}
\DeclareMathOperator{\Imag}{Im}
\DeclareMathOperator{\euler}{e}
\DeclareMathOperator{\cov}{cov}
\DeclareMathOperator{\PolyGrad}{PolyGrad}
\DeclareMathOperator{\gradient}{grad}
\DeclareMathOperator{\rotation}{rot}
\DeclareMathOperator{\divergenz}{div}
\DeclareMathOperator{\imaginär}{j}
\DeclareMathOperator\arccot{arccot}
\DeclareMathOperator\arcsec{arcsec}
\DeclareMathOperator\arcossec{arcossec}
\DeclareMathOperator\arsinh{arsinh}
\DeclareMathOperator\arcosh{arscosh}
\DeclareMathOperator\artanh{artanh}
\DeclareMathOperator\arcoth{arcoth}


%%%%%%%%%%%%%%%%%%%%%%%%%%%%
% Allgemeine Einstellungen %
%%%%%%%%%%%%%%%%%%%%%%%%%%%%
%Pdf Info
\hypersetup{pdfauthor={\authorname},pdftitle={\titleinfo},colorlinks=false}
\author{\authorname}
\title{\titleinfo}


%%%%%%%%%%%%%%%%%%%%%%%
% Kopf- und Fusszeile %
%%%%%%%%%%%%%%%%%%%%%%%
\pagestyle{fancy}
\fancyhf{}
%Linien oben und unten
\renewcommand{\headrulewidth}{0.5pt} 
\renewcommand{\footrulewidth}{0.5pt}

%Kopfzeile links bzw innen
\fancyhead[L]{\titleinfo{ }}
%Kopfzeile mitte
%\fancyhead[C]{}
%Kopfzeile rechts bzw. aussen
\fancyhead[R]{Seite \thepage { }von \pageref{LastPage}}

%Fusszeile links bzw. innen
\fancyfoot[L]{\footnotesize{\authorname}}
%Fusszeile mitte
\fancyfoot[C]{\footnotesize{Elektrotechnik@HSR}}
%Fusszeile rechts bzw. ausen
\fancyfoot[R]{\footnotesize{\today}}
% Einrücken verhindern versuchen
\setlength{\parindent}{0pt}



%%%%%%%%%%%%%%%%%%%%%%%%%%
%	Dokument			%%
%%%%%%%%%%%%%%%%%%%%%%%%%%
\begin{document}
\title{\Huge{Nice to Know}}
\maketitle
Zusammenstellung von Sachen, welche immer wieder gebraucht werden

\tableofcontents
\setcounter{tocdepth}{2} %Setzt tiefe des Inhaltsverzeichnis
\thispagestyle{empty}
\newpage
\input{sections/siEinheiten}
\input{sections/griechischesAlphabet}
\section{Rechengesetze}
\begin{multicols}{2}
\subsection{Potenzregeln}
\renewcommand{\arraystretch}{2}
\begin{tabular}{|c|c|}
	\hline $a^0=1$ & $a^-n= \frac{1}{a^n}$\\
	\hline $a^m \cdot a^n = a^m+n$ & $\frac{a^n}{a^m}= a^n-m$\\
	\hline $(a^n)^m = a^n*m$ & $(\frac{a}{b})^n = \frac{a^n}{b^n}$\\
	 \hline $a^n*b^n = (ab)^n$ & $a^\frac{b}{n}= \sqrt[n]{a^b}$\\
	 \hline
\end{tabular}

\subsection{Wurzelregeln}
\renewcommand{\arraystretch}{2}
\begin{tabular}{|c|c|}
	\hline $\sqrt[n]{a^n}= (\sqrt[n]{a})^n$ & $\sqrt[n]{\frac{a}{b}} = \frac{\sqrt[n]{a}}{\sqrt[n]{b}}$\\
	\hline $\sqrt[n]{a^x}=(\sqrt[n]{a})^x$ & $a\sqrt[n]{x} + b \sqrt[n]{x} = (a+b)\sqrt[n]{x}$\\
	\hline $\sqrt[n]{a \cdot b} = \sqrt[n]{a} \cdot \sqrt[n]{b}$ & $a\sqrt[n]{x} - b \sqrt[n]{x} = (a-b)\sqrt[n]{x}$\\
	\hline $\sqrt[n]{\sqrt[m]{a}}= \sqrt[n \cdot m]{a}$ & $\sqrt[n]{a^x}= a^\frac{x}{n}$ \\
	\hline
\end{tabular}
\end{multicols}

\subsection{Logarithmusregeln}
\begin{minipage}{10cm}
	\begin{tabbing}
		xxxxxxxxxxxxxxxx \= xxxxxxxxxxxxxxxx \= \kill
		$lg(x) = \log_{10} x$ \> $ln(x) = \log_{e} x$ \> $lb(x) = \log_{2} x$
	\end{tabbing}
	\renewcommand{\arraystretch}{2}
	\begin{tabular}{|c|c|}
		\hline $\log{xy} = \log{x} + log{y}$ & $\log{\sqrt[n]{x}}=\log{x^\frac{1}{n}}$\\
		\hline $\log{\frac{x}{y}}= \log{x} + log{y}$ & $\log{x^y}= y\log{x}$ \\
		\hline $\log{\sqrt[n]{x}}= \frac{\log{x}}{n}$ & $\log{1}=0$\\
		\hline
	\end{tabular}
\end{minipage}
\begin{minipage}{5cm}
	\includegraphics[width=4cm]{images/potenz_logarithmus.png}	
\end{minipage}

\begin{multicols}{2}
	\subsection{Binom}
	\renewcommand{\arraystretch}{2}
	\begin{tabular}{|c|}
		\hline $a^2+2ab+b^2 = (a+b)(a+b)$\\
		\hline $a^2-2ab+b^2 = (a-b)(a-b)$\\
		\hline $a^2-b^2= (a+b)(a-b)$\\
		\hline $(a \pm b)^3 =a^3 \pm  3 a^{2} b + 3 a b^2 \pm b^3 $\\
		\hline $(a \pm b)^4 =a^4 \pm  4 a^{3} b + 6a^2b^2 \pm 4 a b^3 +	b^4$\\
		\hline
	\end{tabular}
	
	\subsection{Quadratische Gleichung}
	\begin{tabbing}
		xxxxxxxxxxxxxxxxxxxx \= \kill
		$ax^2+bx+c=0$ \> $x_{1,2} = \dfrac{-b \pm \sqrt{b^2 - 4ac}}{2a}$
	\end{tabbing}
\end{multicols}
\clearpage
\pagebreak
\subsection{Partialbruchzerlegung}
\[f(x)=\frac{x^2+20x+149}{x^3+4x^2-11x-30} \Rightarrow \; \begin{array}{l}\text{Nenner faktorisieren}
\end{array} \Rightarrow
x^{3}+4x^{2}-11x-30=(x+2)(x^{2}+2x-15)=(x+2)(x+5)(x-3)\] Ansatz:
\[f(x)=\frac{x^2+20x+149}{x^3+4x^2-11x-30}=\frac{A}{x-3} + \frac{B}{x+2} + \frac{C}{x+5}=
\frac{A(x+2)(x+5)+B(x-3)(x+5)+C(x-3)(x+2)}{(x-3)(x+2)(x+5)}\]
Gleichungssystem aufstellen mit beliebigen $x_i$-Werten (am Besten Polstellen oder 0,1,-1 wählen):
\[\begin{array}{l}x_1=3:\;-9+60+149=A\cdot5\cdot8\;\;\;\Rightarrow A=5\\
x_2=-2:\;-4-40+149=B(-5)\cdot3\; \Rightarrow B=-7\\
x_3=-5:\;-25-100+149=C(-8)(-3) \Rightarrow C=1 \end{array} \Rightarrow
f(x)=\frac{5}{x-3}-\frac{7}{x+2}+\frac{1}{x+5}\] weitere Ansätze für andere
Typen von Termen: \[f(x)=\frac{5x^2-37x+54}{x^3-6x^2+9x}=\frac{A}{x}+\frac{B}{x-3}+\frac{C}{(x-3)^2}=\frac{A(x-3)^2+Bx(x-3)+Cx}{x(x-3)^2}\]
\[f(x)=\frac{1,5x}{x^3-6x^2+12x-8}=\frac{A}{x-2}+\frac{B}{(x-2)^2}+\frac{C}{(x-2)^3}=\frac{A(x-2)^2+B(x-2)+C}{(x-2)^3}\]
\[f(x)=\frac{x^2-1}{x^3+2x^2-2x-12}=\frac{A}{x-2}+\frac{Bx+C}{x^2+4x+6}=\frac{A(x^2+4x+6)+(Bx+C)(x-2)}{(x-2)(x^2+4x+6)}\]


Variante mit Koeffizientenvergleich: \\
\begin{minipage}{9cm}
	\[F(s) = \frac{1}{s(s^2+6s+13)} = \frac{A}{s} + \frac{Bs+C}{s^2+6s+13}\]
	\[1 = A(s^2+6s+13) + s(Bs+C)\] 
	\[1 = s^2(A+B) + s(C+6A) + 13A\] 
	\[\Rightarrow 1 = 13A; (A+B)=0; (C+6A)=0\]
	\[\Rightarrow A=\frac{1}{13}; B=-\frac{1}{13}; C=-\frac{6}{13}\]
\end{minipage}
\begin{minipage}{9cm}
	$s^2: A+B = 0$\\
	$s^1: 6A+C =0$\\
	$s^0: 13A = 1$
\end{minipage}

\subsection{Hornerschema}
\begin{minipage}[t]{9cm}
	- Pfeile $\Rightarrow$ Multiplikation\\
	- Zahlen pro Spalte werden addiert\\
	\includegraphics[width=6cm]{images/hornerschema_1.png}\\
	$x_1 \Rightarrow$ Nullstelle (muss erraten werden!!)\\
	oberste Zeile = zu zerlegendes Polynom			
\end{minipage}
\begin{minipage}[t]{9cm}
	\textbf{Beispiel:}\\
	$f(x) = x^3-67x-126$\\
	\includegraphics[width=6cm]{images/hornerschema_2.png}\\
	$\Rightarrow f(x) = (x-x_1)(b_2x^2 + b_1x + b_0) = (x+2)(x^2-2x-63)$	
\end{minipage}
\subsection{Winkelmasse}
\begin{tabular}{|l|l|l|}
	\hline & \textbf{Gradmass} & \textbf{Bogenmass}\\
	\hline \textbf{Einheit}& Grad, ° & Radiant, rad\\
	\hline \textbf{Vollwinkel}&  360° & 2$\pi$ rad\\
	\hline \textbf{Umrechnung} & $°= \frac{360}{2\pi} \cdot rad$ & $rad= \frac{2\pi}{360} \cdot °$\\
	\hline
\end{tabular}
\clearpage
\pagebreak
\input{sections/geometrischeGesetze}
\input{sections/komplexeZahlen}
\section{Trigonometrie}
\subsection{Winkelargumente}
\renewcommand{\arraystretch}{1.5}
\begin{tabular}{|c|c|c|c|c|c|c|c|c|c|c|c|c|c|c|c|c|c|c|c|c|}
	\hline \textbf{deg °} & $0$& $30$& $45$& $60$ &$90$&$120$& $135$&$150$&$180$ &$210$&$225$&$240$&$270$&$300$&$315$&$330$\\
	\hline \textbf{rad} & $0$& $\frac{\pi}{6}$&$\frac{\pi}{4}$&$\frac{\pi}{3}$&$\frac{\pi}{2}$&$\frac{2\pi}{3}$&$\frac{3\pi}{4}$&$\frac{5\pi}{6}$ &$\pi$&$\frac{7\pi}{6}$&$\frac{5\pi}{4}$&$\frac{4\pi}{3}$&$\frac{3\pi}{2}$&$\frac{5\pi}{3}$&$\frac{7\pi}{4}$&$\frac{11\pi}{6}$\\ 
	\hline \textbf{sin}&$0$&$\frac{1}{2}$&$\frac{\sqrt{2}}{2}$&$\frac{\sqrt{3}}{2}$&$1$&$\frac{\sqrt{3}}{2}$& $\frac{\sqrt{2}}{2}$&	$\frac{1}{2}$&$0$&$-\frac{1}{2}$ &$-\frac{\sqrt{2}}{2}$ &$-\frac{\sqrt{3}}{2}$ &$-1$&$-\frac{\sqrt{3}}{2}$ &$-\frac{\sqrt{2}}{2}$&$-\frac{1}{2}$\\
	\hline \textbf{cos}&$1$&$\frac{\sqrt{3}}{2}$ &$\frac{\sqrt{2}}{2}$&$\frac{1}{2}$ &$0$&$-\frac{1}{2}$&$-\frac{\sqrt{2}}{2}$&$-\frac{\sqrt{3}}{2}$&$-1$&$-\frac{\sqrt{3}}{2}$&$-\frac{\sqrt{2}}{2}$& $-\frac{1}{2}$&$0$&$\frac{1}{2}$&$\frac{\sqrt{2}}{2}$&$\frac{\sqrt{3}}{2}$\\
	\hline
\end{tabular}

\begin{multicols}{2}
	\subsection{Additionstheoreme}
	$\sin(a \pm b)=\sin(a) \cdot \cos(b) \pm \cos(a) \cdot \sin(b)$\\
	$\cos(a \pm b)=\cos(a) \cdot \cos(b) \mp \sin(a) \cdot \sin(b)$\\	
	$\tan(a \pm b)=\dfrac{\tan(a) \pm \tan(b)}{1 \mp \tan(a) \cdot \tan(b)}$
	
	\subsection{Doppel- und Halbwinkel}	
	$\sin(2a)=2\sin(a)\cos(a)$\\
	$\cos(2a)=\cos^2(a)-\sin^2(a)$\\
	$\cos^2 \left(\frac{a}{2}\right)=\frac{1+\cos(a)}{2} \qquad
	\sin^2 \left(\dfrac{a}{2}\right)=\frac{1-\cos(a)}{2}$

	\subsection{Produkte}
	$\sin(a)\sin(b)=\frac{1}{2}(\cos(a-b)-\cos(a+b))$\\
	$\cos(a)\cos(b)=\frac{1}{2}(\cos(a-b)+\cos(a+b))$\\
	$\sin(a)\cos(b)=\frac{1}{2}(\sin(a-b)+\sin(a+b))$\\	
	
	\subsection{Summe und Differenz}
	$\sin(a)+\sin(b)=2 \cdot \sin \left(\frac{a+b}{2}\right) \cdot
	\cos\left(\frac{a-b}{2}\right)$\\
	$\sin(a)-\sin(b)=2 \cdot \sin \left(\frac{a-b}{2}\right) \cdot
	\cos\left(\frac{a+b}{2}\right)$\\
	$\cos(a)+\cos(b)=2 \cdot \cos \left(\frac{a+b}{2}\right) \cdot
	\cos\left(\frac{a-b}{2}\right)$\\
	$\cos(a)-\cos(b)=-2 \cdot \sin \left(\frac{a+b}{2}\right) \cdot
	\sin\left(\frac{a-b}{2}\right)$\\
	$\tan(a) \pm \tan(b)=\dfrac{\sin(a \pm b)}{\cos(a)\cos(b)}$\\
	
	\subsection{Potenzen}
	$\sin^2(a)+\cos^2(a)=1$\\
	$\sin^2(a)-\cos^2(a)=1-2\sin^2(a)$\\
	$\sin^2(a)=\frac{1}{2}(1-\cos(2a))$\\
	$\cos^2(a)=\frac{1}{2}(1+\cos(2a))$\\
	$\sin^3(a)=\frac{1}{4}(3\sin(a)-\sin(3a))$\\
	$\cos^3(a)=\frac{1}{4}(3\cos(a)-\cos(3a))$\\
\end{multicols}

\subsection{Quadrantenbeziehungen}
\begin{tabbing}
	xxxxxxxxxxxxxxxxxxxxxxxxxxxxxxxxxx \= \kill
	$\sin(-a)=-\sin(a)$ \> $\cos(-a)=\cos(a)$\\
	$\sin(\pi - a)=\sin(a)$ \> $\cos(\pi - a)=-\cos(a)$\\
	$\sin(\pi + a)=-\sin(a)$ \> $\cos(\pi +a)=-\cos(a)$\\
	$\sin\left(\frac{\pi}{2}-a \right)=\sin\left(\frac{\pi}{2}+a \right)=\cos(a)$ \>
	$\cos\left(\frac{\pi}{2}-a \right)=-\cos\left(\frac{\pi}{2}+a \right)=\sin(a)$  
\end{tabbing}

\subsection{Plots}
\begin{multicols}{3}
	\subsubsection{Sinus-Funktion}
	\includegraphics[width=5.5cm]{images/sin.png}
	\subsubsection{Cosinus-Funktion}
	\includegraphics[width=5.5cm]{images/cos.png}
	\subsubsection{Tangens-Funktion}
	\includegraphics[width=5.5cm]{images/tan.png}
\end{multicols}
\clearpage
\pagebreak
\input{sections/matrizen}
\section{Wahrscheinlichkeit}
%TODO Wahrscheinlichkeiten hinzufügen
\clearpage
\pagebreak
\section{Differentialrechnung}
\subsection{Differentation von Funktionen einer Variablen}
Der Differentialquotient oder Ableitung einer Funktion beschriebt die Steigung einer Tangente an die Funktion.\\
\begin{equation*}
f'(x)=\lim\limits_{\triangle x \to 0}\frac{f(x + \triangle x)-f(x)}{\triangle x}
\end{equation*}
Die Ableitung einer Funktion $y=f(x)$ ist eine Funktion von x welche mit den Symbolen: $y'$, $\dot{y}$, $Dy$ dargestellt wird.\\
\includegraphics[width=5cm]{images/differential.png}
 
\subsection{Ableitungsregeln}
\renewcommand{\arraystretch}{1.5}
\begin{tabular}{|l|l|}
	\hline \textbf{Konstantenregeln}& $c'=0'$\\
	\hline \textbf{Faktorenregeln}& $(cu)'=c u'$\\
	\hline \textbf{Summenregel}& $(u\pm v)'= u' \pm v'$\\
	\hline \textbf{Produktregel}& $(uv)'=u'v + uv'$\\
	\hline \textbf{Quotientenregel}& $(\frac{u}{v})'= \frac{u'v-uv'}{u^{2}}$\\
	\hline \textbf{Kettenregel}& $y=u(v(x)) ; y'=\frac{du}{dv} \frac{dv}{dx}$\\
	\hline \textbf{Potenzregel} & $(u^{a})'=au^{a-1}$\\
								& $(\frac{1}{u})'= \frac{u'}{u^2}$\\
	\hline	\textbf{Wurzelregel} & $f(x)=\sqrt{x} ; f'(x)=\frac{1}{2\sqrt{x}}$\\
	\hline	\textbf{Logarithmusregel} & $\ln{u}'=\frac{u'}{u}$\\ 
	\hline	\textbf{Differentation der Umkehrfunktion} & $(f^{-1})'(y)=\frac{1}{f'(x)} $\\
	\hline 
\end{tabular}
\includegraphics[width=5cm]{images/sin_cos.png}
\newpage
\subsection{Ableitungen elementarer Funktionen}
\renewcommand{\arraystretch}{2.2}
\begin{tabular}{|l|l||l|l|}
	\hline
	\textbf{Funktion} & \textbf{Ableitung} & \textbf{Funktion} &
	\textbf{Ableitung}\\\hline
	\hline $C$ (Konstante) & 0 & $\sec x$ & $\dfrac{\sin x}{\cos^2 x}$ \\
	\hline $x$ & 1 & $\sec^{-1} x$ & $\dfrac{-\cos x}{\sin^2 x}$\\
	\hline $x^n$ ($n\in\mathbb{R}$) & $nx^{n-1}$ & $\arcsin x \quad (|x| < 1)$ &
	 $\dfrac{1}{\sqrt{1-x^2}}$\\
	\hline $\dfrac{1}{x}$ & $-\dfrac{1}{x^2}$ & $\arccos x \quad (|x| < 1)$ &
	$-\dfrac{1}{\sqrt{1-x^2}}$\\
	\hline $\dfrac{1}{x^n}$ & $-\dfrac{n}{x^{n+1}}$ & $\arctan x$ & $\dfrac{1}{1+x^2}$\\
	\hline $\sqrt{x}$ & $\dfrac{1}{2\sqrt{x}}$ & $\arccot{x} $ & $-\dfrac{1}{1+x^2}$\\
	\hline $\sqrt[n]{x}\quad (n\in\mathbb{R}, n \neq 0, x > 0)$ &
	$\dfrac{1}{n\sqrt[n]{x^{n-1}}}$ & $\arcsec x$ & $\dfrac{1}{x\sqrt{x^2-1}}$\\
	\hline $\mathrm{e}^x$ & $\mathrm{e}^x$ & $\arcossec x$ & $-\dfrac{1}{x\sqrt{x^2-1}}$\\
	\hline $\mathrm{e}^{bx}\quad (b\in\mathbb{R})$ & $b\mathrm{e}^{bx}$ & $\sinh x$ &
	$\cosh x$\\
	\hline $a^x\quad (a > 0)$ & $a^x\ln a$ & $\cosh x$ & $\sinh x$\\
	\hline $a^{bx}\quad (b\in\mathbb{R}, a > 0)$ & $ba^{bx}\ln a$ & $\tanh x$ &
	$\dfrac{1}{\cosh^2 x}$\\
	\hline $\ln x$ & $\dfrac{1}{x}$ & $\coth x \quad(x \neq 0)$ & $-\dfrac{1}{\sinh^2 x}$\\
	\hline $\log_a{x} \quad (a > 0, a \neq 1, x > 0)$ &
	$\dfrac{1}{x}\log_a{\mathrm{e}}=\dfrac{1}{x\ln a}$ & $\arsinh x$ &
	$\dfrac{1}{\sqrt{1+x^2}}$\\
	\hline $\lg x \quad (x > 0)$ & $\dfrac{1}{x}\lg \mathrm{e}\approx \dfrac{0.4343}{x}$
	& $\arcosh x \quad (x > 1)$ & $\dfrac{1}{\sqrt{x^2-1}}$\\
	\hline $\sin x$ & $\cos x$ & $\artanh x \quad (|x| < 1)$ & $\dfrac{1}{1-x^2}$\\
	\hline $\cos x$ & $-\sin x$ & $\arcoth x \quad (|x| > 1)$ & $-\dfrac{1}{x^2-1}$\\
	\hline $\tan x \quad (x\neq(2k+1)\dfrac{\pi}{2}, k\in\mathbb{Z})$ & $\dfrac{1}{\cos^2
		x}=\sec^2 x$ & $[f(x)]^n \quad (n\in\mathbb{R})$ & $n[f(x)]^{n-1}f'(x)$\\
	\hline $\cot x \quad (x\neq k\pi, k\in\mathbb{Z})$ & $\dfrac{-1}{\sin^2 x}=-cosec^2x$ & $\ln f(x) \quad (f(x)> 0)$ & $\dfrac{f'(x)}{f(x)}$\\
	\hline
\end{tabular}
\newpage
\section{Integralrechung}
Die Integralrechnung ist die Umkehrung der Differentialrechnung. Bei der Differentialrechnung wird zu einer gegebenen Funktion $f(x)$ die Ableitung $f'(x)$ bestimmt. Bei der Integralrechung wird zu eine Ableitung $f'(x)$ eine Funktion $f(x)$ gesucht welche mit der Ableitung übereinstimmt. 
\begin{equation*}
	F'(x)=\frac{dF}{dx}=f(x)
\end{equation*}
\begin{tabular}{|c|c|}
	\hline \textbf{Bestimmtes Integral} & \textbf{Unbestimmtes Integral}\\
	\hline $\int_{a}^{b}{f(x)dx}$ & $\int{f(x)dx}=F(x)+C$\\
	\hline \tabbild[width=4cm]{images/best_integral.png}& \tabbild[width=4cm]{images/unb_integral.png}\\
	\hline
\end{tabular}

\subsection{Integrationsregeln}
\begin{tabular}{|l|l|l|}
	\hline	\textbf{Integrationskonstane} & $\int{f(x)dx}=F(x)+C$&\\
	\hline	\textbf{Faktorregel}& $\int{af(x)dx}=a \int{f(x)dx}$&\\
	\hline	\textbf{Summenregel}& $\int{[u(x) \pm v(x)]dx}=\int{u(x)dx} \pm \int{v(x)dx}$&\\
	\hline \textbf{Potenzregel}& $\int {f'(x)\cdot f(x)^{a}}=\frac{f(x)^{a+1}}{a+1}+C$& $\int {\sin^3{(x)} \cdot \cos(x)}=\frac{\sin^4{(x)}}{4}+C$\\
	\hline \textbf{Logarithmusregel} & $\int{\frac{f'(x)}{f(x)}dx}= \ln{(x)}+C$&$\int{\frac{x^2}{1+x^3}}=\ln(1+x^3)+C$ \\
	\hline \textbf{Linearität} &$\int{f(ax \pm b)dx}=\frac{1}{a} \int{f(x)dx} \pm \int{b dx}$&\\
	\hline \textbf{Partielle Integration} &$\int{u(x)v'(x)dx}=u(x)v(x)- \int{u'(x)v(x)dx}$ &$\int{ \underbrace{x^2}_{v'} \cdot \underbrace{\ln{x}}_{u}}=\frac{x^3}{3} \cdot \ln{x} - \int{\frac{x^3}{3} \cdot \frac{1}{x}}$\\
			\textbf{(Produktregel)} & &\\
	\hline	\textbf{Substitution} &$\int_{a}^{b}{f(x)dx}=\int_{g(a)}^{g(b)}{f(g(t))\cdot g'(t)dt}$  &$\int{(x^2+2)^3 \cdot 2xdx}\;\;//\;u=x^2+2$\\
	& &$\int{u^3 \cdot 2xdx}\;\;//\;du=2xdx$\\
	& &$\int{u^3 \cdot du}=\frac{u^4}{4}=\frac{(x^2+2)^4}{4}$\\
	\hline
\end{tabular}

\subsection{Wichtige Integrale}
\renewcommand{\arraystretch}{2}
\begin{tabular}{|l|l|}
	\hline
	$\int \sin(x)dx=-\cos(x)$ & $\int \sin(a+bx)dx=-\frac1b \cos(a+bx)$\\
	\hline
	$\int \sin^2(x)dx=-\frac14 \sin(2x)+\frac x2$ 
	& $\int
	e^{ax+c}\sin(bx+d)dx=\frac{e^{ax+c}}{a^2+b^2}(a\sin(bx+d)-b\cos(bx+d))$\\
	\hline
	$\int \cos(x)dx=\sin(x)$ & $\int \cos(a+bx)dx=\frac1b \sin(a+bx)$\\
	\hline
	$\int \cos^2(x)dx=\frac14 \sin(2x)+\frac x2$ 
	& $\int
	e^{ax+c}\cos(bx+d)dx=\frac{e^{ax+c}}{a^2+b^2}(a\cos(bx+d)+b\sin(bx+d))$\\
	\hline
	$\int e^x dx=e^x$ & $\int e^{ax}dx=\frac1a e^{ax}$\\
	\hline
	$\int xe^{ax}dx=\frac{1}{a^2} e^{ax}(ax-1)$ & $\int x^2 e^{ax} dx =
	e^{ax}\left( \frac{x^2}{a} - \frac{2x}{a^2} + \frac{2}{a^3}\right)$ \\
	\hline
	$\int x^n e^{ax} dx = \frac{1}{a} x^n e^{ax} - \frac{n}{a} \int x^{n-1}
	e^{ax} dx$ & \\
	\hline
\end{tabular}
\clearpage
\pagebreak
\section{Fourierreihen}
Mithilfe der Fourierreihe kann ein beliebiges periodisches Signal in seine Grundschwingungen (Harmonische) aufgeteilt werden.\\
%\renewcommand{\arraystretch}{2}
\begin{tabular}{|l|l|}
	\hline \textbf{Fourierreihe} &${f(t) = \frac{a_0}{2} + \sum\limits_{k=1}^{\infty} \left[a_k \cos(k
		\omega t) + b_k \sin(k \omega t)\right]} \rightarrow FR[f(t)] \; \omega=\frac{2 \pi}{T}=2 \pi$\\
	\hline \textbf{Berechnung Reell} & $a_0 = \frac{2}{T}\int\limits_0^{T}
	f(t)dt, \quad a_k = \frac{2}{T}\int\limits_0^{T} f(t)\cos(k \omega_1 t) dt, \quad b_k =
	\frac{2}{T}\int\limits_0^{T} f(t)\sin(k \omega_1 t) dt$\\
	\hline \textbf{Berechnung komplex} & $c_k=\frac{1}{T}\int_0^T{f(t)}\cdot
		e^{-jk\omega t}, \quad f(t) = \sum\limits_{k = -\infty}^{\infty} c_k \cdot e^{j k \omega t dt}$\\
	\hline
\end{tabular}

\subsection{Symmetrie}
\begin{tabular}{|l|l|l|l|}
	\hline
	\textbf{gerade Funktion} & \textbf{ungerade Funktion} &
	\textbf{Halbperiode 1} & \textbf{Halbperiode 2}\\
	\hline
	\tabbild[width=3cm]{images/gerade_funktion.png}&
	\tabbild[width=3cm]{images/ungerade_funktion.png}&   
	\tabbild[width=3cm]{images/halbperiode_1.png}&   
	\tabbild[width=3cm]{images/halbperiode_2.png}\\
	\hline $f(-t)=f(t)$ & $f(-t)=-f(t)$ & $f(t)=f(t+\pi)$ & $f(t)=-f(t+\pi)$\\
	$b_k=0$ & $a_k=0$ & $a_{2k+1}=0$ & $a_{2k}=0$\\
	$a_k = \frac{4}{T} \int\limits_0^{\frac{T}{2}} f(t) \cdot \cos(k \omega_1
	t) dt$ &
	$b_k =  \frac{4}{T} \int\limits_0^{\frac{T}{2}} f(t) \cdot
	\sin(k \omega_1 t) dt$ &
	$b_{2k+1}=0$ & $b_{2k}=0$\\
	\hline $\Im(c_n)=0$ &$\Re(c_n)$ & &\\ 
	\hline
\end{tabular}

\subsection{Wichtige Fouriereihen}
\begin{tabular}{|l|l|l|}
	\hline \textbf{Dreieckfunktion} & \textbf{Rechteckfunktion}& \textbf{Impulsfunktion}\\
	\hline \tabbild[width=4cm]{images/dreiecksig.png} &\tabbild[width=4cm]{images/rechteck.png}&\tabbild[width=4cm]{images/impuls.png}\\
	\hline $a_0=A$& $a_0=0$& $a_0=\frac{2At_1}{T}$\\
	\hline $a_k=-\frac{4A}{\pi^{2}k^{2}}$&$a_k=0$&$a_k=\frac{A}{\pi k}(\sin(\frac{2 \pi t_1}{T}k))$\\
	\hline $b_k=0$&$b_k=\frac{4A}{\pi k}$&$b_k=-\frac{A}{\pi k}(1-\cos(\frac{2 \pi t_1}{T}k))$\\
	\hline
\end{tabular}
\newpage
\input{sections/fouriertransformation}
\section{Laplacetransformation}
Mittels der Laplacetransforamtion können kausale Signale analysiert werden. Sollte das Signal nicht kausal sein wird es mit der Einschaltfunktion multipliziert.\\
\begin{tabular}{c}
	$F(s)=\int\limits_0^\infty f(t)e^{-st}dt \qquad s=\sigma+j\omega$\\
	$Originalbereich \;\laplace\; Bildbereich$\\
\end{tabular}

\subsection{Eigenschaften der Laplacetransformation}
\begin{tabular}{|l|l|}
	\hline
	Linearität & 
	$\alpha\cdot f(t) + \beta\cdot g(t) \;\laplace\; \alpha\cdot F(s) + \beta\cdot
	G(s)$ \\
	\hline
	Zeitskalierung &
	$f(\alpha t) \;\laplace\; \frac{1}{\alpha}F \left (\frac{s}{\alpha} \right ) \quad 0
	<\alpha \in\mathbb{R}$ \\
	\hline
	Faltung im Zeitbereich &
	$f(t) \ast g(t) = \int\limits_{0}^{\infty} f(\tau)g(t-\tau)d\tau \;\laplace\; F(s)
	\cdot G(s)$\\
	\hline
	Faltung im Frequenzbereich &
	$f(t) \cdot g(t) \;\laplace\; \frac{1}{2\pi j}\int\limits_{c-j\infty}^{c+j\infty}
	F(\xi) G(s-\xi)d\xi$ \\
	\hline
	Ableitung im Zeitbereich &
	$\frac{\partial f(t)}{\partial t} \;\laplace\; sF(s)
	-f(0+)$ \\
	\hline
	Ableitungen im Zeitbereich &
	$\frac{\partial^n f(t)}{\partial t^n} \;\laplace\; s^nF(s)
	-s^{n-1}f(0+)-s^{n-2}\frac{\partial f(0+)}{\partial t}-\ldots
	-s^0\frac{\partial^{n-1} f(0+)}{\partial t^{n-1}}$ \\
	\hline
	Multiplikation mit $t$ &
	$t\cdot f(t)  \;\laplace\; \frac{-\partial F(s)}{\partial s}$ \\
	\hline
	Ableitung im Frequenzbereich &
	$(-t)^n f(t) \;\laplace\;  \frac{\partial^n F(s)}{\partial s^n}$ \\
	\hline
	Verschiebung im Zeitbereich &
	$f(t\pm t_0) \;\laplace\; F(s)e^{\pm t_0 s}$ \\
	\hline
	Verschiebung im Frequenzbereich &
	$f(t)e^{\mp\alpha t} \;\laplace\; F(s\pm\alpha)$ \\
	\hline
	Integration &
	$\int\limits_0^t f(\tau)d\tau \;\laplace\; \frac{F(s)}{s}$ \\
	\hline
	Anfangswert &
	$\lim_{t\rightarrow 0} f(t) = \lim_{s\rightarrow \infty} sF(s),\text{~wenn
	}  \lim_{t\rightarrow 0} f(t)\text{~existiert}.$ \\
	\hline
	Endwert &
	$\lim_{t\rightarrow \infty} f(t) = \lim_{s\rightarrow 0} sF(s),\text{~wenn
	}  \lim_{t\rightarrow \infty} f(t)\text{~existiert}.$ \\
	\hline
\end{tabular}
\newpage
\subsection{Rücktransformation}
\subsubsection{Vorgehen}
\begin{minipage}{12cm}
	\begin{enumerate}
		\item Benutzung einer Tabelle zugehöriger Original-, und Bildfunktionen (Korrespondenzen)
		\item Umformen (Kürzen, Erweitern, etc.) um auf Korrespondenz zu schliessen
		\item Mittels Partialbruchzerlegung auf Korrespondenz schliessen
	\end{enumerate}
\end{minipage}
\begin{minipage}{7cm}
	\includegraphics[width=7cm]{images/IntTra.jpg}
\end{minipage}

\subsubsection{Laplacetabelle}
\begin{multicols}{2}
	\begin{center}
		\begin{tabular}{|lcc|}
			\hline
			$\sigma \left( t \right)$ & $\; \laplace \;$ & $\frac{1}{s}$ \\
			$\sigma \left( t \right) \cdot t$ & $\; \laplace \;$ & $\frac{1}{s^2}$\\
			$\sigma \left( t \right) \cdot t^2$ & $\; \laplace \;$ & $\frac{2}{s^3}$\\
			$\sigma \left( t \right) \cdot t^n$ & $\; \laplace \;$ & $\frac{n!}{s^{n+1}}$\\
			$\sigma \left( t \right) \cdot e^{\alpha t}$ & $\; \laplace \;$ &
			$\frac{1}{s-\alpha}$\\
			$\sigma \left( t \right) \cdot t \cdot e^{\alpha t}$ & $\; \laplace \;$ &
			$\frac{1}{( s - \alpha )^2}$\\
			\hline
		\end{tabular}
	\end{center}
	\columnbreak
	\begin{center}
		\begin{tabular}{|lcc|}
			\hline
			$\sigma \left( t \right)\cdot t^2 \cdot e^{\alpha t}$ &
			$\; \laplace \;$ & $\frac{2}{{( s - \alpha )}^3}$\\
			$\sigma \left( t \right)\cdot t^n \cdot e^{ \alpha t}$ &
			$\; \laplace \;$ & $\frac{n!}{(s-\alpha)^{n+1}}$\\
			$\sigma \left( t \right) \cdot \sin \left(\omega t \right)$ & $\; \laplace \;$ &
			$\frac{\omega}{s^2 + {\omega^2}}$\\
			$\sigma \left( t \right) \cdot \cos \left( \omega t \right)$ & $\; \laplace \;$ &
			$\frac{s}{ s^2 + \omega^2}$\\
			$\delta \left( t \right)$ & $\; \laplace \;$ & $1\left( s \right)$ \\
			$\delta \left( t - \alpha \right)$ & $\; \laplace \;$ & $e^{- \alpha s}$\\
			\hline
		\end{tabular}
	\end{center}
\end{multicols}

\subsection{Lösen von Differentialgleichungen mit Laplace}
\begin{minipage}{11.5cm}
	\begin{tabular}{| l | l |}
		\hline
		Übertragungsfunktion & $G(s) = \frac{1}{p(s)} \quad g(t) \; \laplace \; G(s)$\\
		\hline Charakteristisches Polynom & $p(s)$\\
		\hline
		Frequenzgang & $G(j\omega) = H(\omega)$ \\
		\hline
		Impulsantwort & $y_{\delta}(t) = g(t) = y_{\sigma}'(t) \; \laplace \; G(s) = \frac{1}{p(s)}=Y_{\delta}(s)$\\
		\hline
		Sprungwantwort & $y_{\sigma}(t)=\int\limits_0^t g(u) du \; \laplace \; \frac{G(s)}{s} = \frac{1}{s \cdot p(s)} = Y_{\sigma}(s)$\\
		\hline
		Eigenschwingung & $\frac{h(s)}{p(s)}$ \\
		\hline
		äussere Erregung & $\frac{F(s)}{p(s)}$ \\
		\hline
		stationärer Zustand & = ungedämpfte Eigenschwingung\\
		\hline
	\end{tabular}
\end{minipage}
\newpage
\subsubsection{Lineare DGL mit Anfangswerten}
Gegeben sei eine Differentialgleichung mit Anfangsbedingungen.Sind keine Anfangsbedingungen vorhanden können die daraus entstehenden Terme vernachlässigt werden.\\
\vspace{2pt}
\begin{tabular}{l l}
	DGL & $a_n y^{n}(t)+a_n-1 y^{n-1}+...a_1 y'(t)+a_0 y(t)=f(t)$\\
	Endterm & $a_0\cdot[Y(s)]+a_1 \cdot [sY(s)-f(0)]+a_n-1y^n-1 \cdot [s^nY(s)-s^{n-1} \cdot f(0)-s^{n-2}f'(0)...-f^{n-1}(0)]=F(s)$\\
\end{tabular}

\begin{minipage}{15cm}
	$y(t) \; \laplace \;  Y(s)$\\
	$y'(t) \; \laplace \; sY(s) - f(0)$\\
	$y''(t) \; \laplace \; s^2Y(s)-s\cdot f(0) - f'(0)$\\
	$y'''(t) \; \laplace \; s^3Y(s)-s^2\cdot f(0)-s \cdot f'(0) - f''(0)$\\
	$y^n(t) \; \laplace \; s^nY(s)-s^{n-1}\cdot f(0)-s^{n-2} \cdot f'(0) ...- f^{n-1}(0)$\\
	$y^{(n)} \; \laplace \; 
	\underbrace{s^nY(s)}_{Y(s) \cdot p(s)}
	\underbrace{-s^{n-1}y_0 - \dots - y^{(n-1)}}_{h(s)}$\\
\end{minipage}
\includepdf[pages={8-9},pagecommand={},,scale=0.98]{sections/Tabellen.pdf}
\clearpage
\pagebreak
\input{sections/logischeOperationen}
\input{sections/nassiShneidermann}
\section{Signale}
\subsection{Harmonische Schwingungen}
Als harmonische Schwingung bezeichnet man eine sinusförmige Schwingung. 
\begin{equation*}
x(t) \underbrace{A}_{Amplitude} \cdot \sin(\underbrace{\omega}_{Kreisfrequenz}t +\underbrace{\phi}_{Phasenverschiebung})+\underbrace{c}_{DC-Anteil}
\end{equation*}
\begin{tabular}{|l|l|l|}
	\hline
	\textbf{Ordnung n}& \textbf{Frequenz}& \textbf{Name der Komponente}\\
	\hline $0$& $0$& Gleichstromanteil\\
	\hline $1$& $f_0$& Grundwelle / 1.Harmonische\\
	\hline $2$& $2f_0$& 1.Oberwelle / 2.Harmonische\\
	\hline $3$& $3f_0$& 2.Oberwelle/ 3.Harmonische\\
	\hline $n$& $nf_0$& n-1.Oberwelle / n-te .Harmonische\\
	\hline
\end{tabular}
\clearpage
\pagebreak
%TODO Fehler font beheben
\subsection{Logarithmische Darstellungen}
\renewcommand{\arraystretch}{1.2}
\begin{tabular}{ll}
	\parbox{7cm}{
		\scriptsize
		\begin{tabular}{|c|c|c|c|}
			\hline
			\textbf{Lrel. (dB)} & \textbf{Lrel. (NP)} & \textbf{P2/P1} & \textbf{A2/A1} \\ \hline
			$100.000$ & $11.513$ & $10^{10}$ & $10^5$ \\ \hline
			$90.000$ & $10.362$ & $10^9$ & $31622.777$ \\ \hline
			$80.000$ & $9.210$ & $10^8$ & $10^4$ \\ \hline
			$70.000$ & $8.059$ & $10^7$ & $3162.278$ \\ \hline
			$60.000$ & $6.908$ & $10^6$ & $10^3$ \\ \hline
			$50.000$ & $5.756$ & $10^5$ & $316.228$ \\ \hline
			$40.000$ & $4.605$ & $10^4$ & $10^2$ \\ \hline
			$30.000$ & $3.454$ & $10^3$ & $31.623$ \\ \hline
			\textbf{$20.000$} & $2.303$ & \textbf{$10^2$} & \textbf{$10.000$} \\ \hline
			$19.085$ & $2.197$ & $81.000$ & $9.000$ \\ \hline
			$19.000$ & $2.187$ & $79.433$ & $8.913$ \\ \hline
			$18.062$ & $2.079$ & $64.000$ & $8.000$ \\ \hline
			$18.000$ & $2.072$ & $63.096$ & $7.943$ \\ \hline
			$17.000$ & $1.957$ & $50.119$ & $7.079$ \\ \hline
			$16.902$ & $1.946$ & $49.000$ & $7.000$ \\ \hline
			$16.000$ & $1.842$ & $39.811$ & $6.310$ \\ \hline
			$15.563$ & $1.792$ & $36.000$ & $6.000$ \\ \hline
			$15.000$ & $1.727$ & $31.623$ & $5.623$ \\ \hline
			$14.000$ & $1.612$ & $25.119$ & $5.012$ \\ \hline
			\textbf{$13.979$} & $1.609$ & \textbf{$25.000$} & \textbf{$5.000$} \\ \hline
			$13.000$ & $1.497$ & $19.953$ & $4.467$ \\ \hline
			\textbf{$12.041$} & $1.386$ & \textbf{$16.000$} & \textbf{$4.000$} \\ \hline
			\textbf{$12.000$} & $1.382$ & $15.849$ & $3.981$ \\ \hline
			$11.000$ & $1.266$ & $12.589$ & $3.548$ \\ \hline
			\textbf{$10.000$} & $1.151$ & \textbf{$10.000$} & $3.162$ \\ \hline
			$9.542$ & $1.099$ & $9.000$ & $3.000$ \\ \hline
			$9.000$ & $1.036$ & $7.943$ & $2.818$ \\ \hline
			$8.000$ & $0.921$ & $6.310$ & $2.512$ \\ \hline
			$7.000$ & $0.806$ & $5.012$ & $2.239$ \\ \hline
			\textbf{$6.021$} & \textbf{$0.693$} & \textbf{$4.000$} & \textbf{$2.000$} \\ \hline
			$6.000$ & $0.691$ & $3.981$ & $1.995$ \\ \hline
			$5.000$ & $0.576$ & $3.162$ & $1.778$ \\ \hline
			$4.000$ & $0.461$ & $2.512$ & $1.585$ \\ \hline
			\textbf{$3.010$} & \textbf{$0.347$} & \textbf{$2.000$} & \textbf{$1.414$} \\ \hline
			$3.000$ & $0.345$ & $1.995$ & $1.413$ \\ \hline
			$2.000$ & $0.230$ & $1.585$ & $1.259$ \\ \hline
			$1.000$ & $0.115$ & $1.259$ & $1.122$ \\ \hline
			$0.000$ & $0.000$ & $1.000$ & $1.000$ \\ \hline
			-$1.000$ & -$0.115$ & $0.794$ & $0.891$ \\ \hline
			-$2.000$ & -$0.230$ & $0.631$ & $0.794$ \\ \hline
			-$3.000$ & -$0.345$ & $0.501$ & $0.708$ \\ \hline
			-$4.000$ & -$0.461$ & $0.398$ & $0.631$ \\ \hline
			-$5.000$ & -$0.576$ & $0.316$ & $0.562$ \\ \hline
			-$6.000$ & -$0.691$ & $0.251$ & $0.501$ \\ \hline
			-$7.000$ & -$0.806$ & $0.200$ & $0.447$ \\ \hline
			-$8.000$ & -$0.921$ & $0.158$ & $0.398$ \\ \hline
			-$9.000$ & -$1.036$ & $0.126$ & $0.355$ \\ \hline
			-$10.000$ & -$1.151$ & $0.100$ & $0.316$ \\ \hline
			-$15.000$ & -$1.727$ & $0.032$ & $0.178$ \\ \hline
			-$20.000$ & -$2.303$ & $10^{-2}$ & $0.100$ \\ \hline
			-$30.000$ & -$3.454$ & $10^{-3}$ & $0.032$ \\ \hline
			-$40.000$ & -$4.605$ & $10^{-4}$ & $0.010$ \\ \hline
			-$50.000$ & -$5.756$ & $10^{-5}$ & $0.003$ \\ \hline
			-$60.000$ & -$6.908$ & $10^{-6}$ & $0.001$ \\ \hline
			-$70.000$ & -$8.059$ & $10^{-7}$ & $0.000$ \\ \hline
			-$80.000$ & -$9.210$ & $10^{-8}$ & $10^{-4}$ \\ \hline
			-$90.000$ & -$10.362$ & $10^{-9}$ & $3.162 \cdot 10^{-5}$ \\ \hline
			-$100.000$ & -$11.513$ & $10^{-10}$ & $10^{-5}$ \\ \hline
		\end{tabular}
	}
	& \parbox{11.5cm}{
		\normalsize
		Verstärkungsmass L in \textbf{Dezibel} (dB):\\
		$L_P = 10 \cdot \log \left(\frac {P_2} {P_1}\right) \qquad$ Index P: Leistung \\
		$L_A = 20 \cdot \log \left(\frac {A_2} {A_1}\right) \qquad$ Index A: Amplitude \\ 
		
		Dezibel L zu linear: \\
		$P_2 = P_1 \cdot 10^{\frac{L_P}{10}} $ \\
		$A_2 = A_1 \cdot 10^{\frac{L_A}{20}} $ \\
		
		Verstärkungsmass L in \textbf{Neper} (Np):\\
		$L_P = \frac {1}{2} \cdot \ln \left(\frac {P_2} {P_1}\right)$\\
		$L_A = \ln \left(\frac {A_2} {A_1} \right)$ \\
		
		Neper zu linear: \\
		$P_2 = P_1 \cdot e^{2 L_P}$ \\
		$A_2 = A_1 \cdot e^{L_A}$ \\
		
		Die Umrechnung zwischen {\bf dB} und {\bf Np} ist linear: \\
		$1\mbox{~dB} = \frac {\ln(10)} {20} \mbox{~Np} = 0.1151\mbox{~Np}$ \\
		$1\mbox{~Np} = 20 \cdot \log(\mbox{e}) \mbox{~dB} = 8.686\mbox{~dB}$ \\ 
		\\
		Anstatt $\frac{X_2}{X_1}$ für Verstärkungsmasse ($L$) können auch
		$\frac{X_1}{X_2}$ für \textbf{Dämpfungsmasse ($\bf{a}$)} verwendet werden!
		
		\small{($P$ für Leistungen, $A$ für Amplituden)}
		\\ 
		
		\textbf{Hilfen zur Berechnung}\\
		\begin{tabular}{|l|ll|}
			\hline
			$x dB$	& $T_P=P_2/P_1$ &$T_A=A_2/A_1$ \\
			\hline
			$-x dB$	& $1/T_P = D_P$	& $1/T_A = D_A$\\
			$x+3dB$	& $T_P \cdot 2$	& $T_A \cdot \sqrt{2} \approx T_A \cdot 1.414$ \\
			$x+6dB$ & $T_P \cdot 4$ & $T_A \cdot 2$ \\
			$x+10dB$	& $T_P \cdot 10$ & $T_A \cdot \sqrt{10} \approx T_A \cdot 3.162$\\
			\hline
		\end{tabular}
		\begin{tabular}{lllll}
			$T$: & Verstärkungsfaktor & &
			$D$: & Dämpfungsfaktor
		\end{tabular}
		\\ \\
		
		\textbf{Relative Pegel}\\
		\begin{tabular}{|l|l|}
			\hline
			dBu & Spannungspegel bezogen auf 774.6~mV (1~mW an $600\Omega$)\\
			\hline
			\hline
			dBV & Spannungspegel bezogen auf 1~V\\
			\hline
			dB$\mu$V & Spannungspegel bezogen auf 1~$\mu$V\\
			\hline
			dBW & Leistungspegel bezogen auf 1~W\\
			\hline
			dBm & Leistungspegel bezogen auf 1~mW\\
			\hline
		\end{tabular}		
	}
\end{tabular}
\newpage

\subsection{Signalarten}
\begin{tabular}{|c|c|c|c|}
	\hline \textbf{Energiesignal} & \textbf{Leistungsignal} & \textbf{Aperiodisch}& \textbf{Periodisch}\\
	Zeitlich begrenzt& Zeitlich unbegrenzt & &\\
	$E=\int_{-\infty}^{+ \infty}{|x(t)|^2dt}< \infty$&$P=lim_{T \to \infty} \frac{1}{T} \cdot \int_{-\frac{T}{2}}^{+ \frac{T}{2}}{|x(t)|^2dt}= \infty$ & $x(t) \neq x(t+n \cdot T)$& $x(t)=x(t+n \cdot T)$\\
	\hline	\tabbild[width=4cm]{images/energiesignal.png} & \tabbild[width=4cm]{images/leistungssignal.png} & \tabbild[width=4cm]{images/aperiodisch.png}&\tabbild[width=4cm]{images/periodisch.png}\\
	\hline
\end{tabular}
\vspace{3pt}
\begin{tabular}{|c|c|c|c|}
	\hline \textbf{Deterministisch} & \textbf{Stochastisch} & \textbf{Zeitkontinuierlich}& \textbf{Zeitdiskret}\\
	$x(t)=f(t)$& $x(t)=?$ &$x(t)$ ist für Verlauf definiert&$x(t)$ ist nur an\\ & & & Abtastpunkten definiert\\
	\hline	\tabbild[width=4cm]{images/determenistisch.png} & \tabbild[width=4cm]{images/stochastisch.png} & \tabbild[width=4cm]{images/zeitkontinuierlich.png}&\tabbild[width=4cm]{images/zeitdiskret.png}\\
	\hline
\end{tabular}
\vspace{3pt}
\begin{tabular}{|c|c|c|c|}
	\hline \textbf{Amplitudenkontinuierlich} & \textbf{Quantisiert} & \textbf{Analog}& \textbf{Digital}\\
	 $x(t)=y$& $x(t)=y_k$ & &\\
	\hline	\tabbild[width=4cm]{images/amplitudenkontinuierlich.png} & \tabbild[width=4cm]{images/quantisiert.png} & \tabbild[width=4cm]{images/analog.png}&\tabbild[width=4cm]{images/digital.png}\\
	\hline
\end{tabular}

\begin{sidewaystable}
	\subsection{Eigenschaften unterschiedlicher Schwingungsformen}
		\begin{tabular}{|l|c|c|c|c|c|c|c|c|}
			\hline
			Schwingungsform & Funktion & Gleichrichtwert & Formfaktor &
			Effektivwert & Scheitelfaktor & $X_0$ & $X^2$ & var(X) \\
			\hline
			Formel &
			&
			$\overline{\left|x\right|} = \frac1T\int_{0}^{T}\left| x(t)\right|dt$&
			$\frac{X}{\overline{\left|x\right|}}$&
			$X = \sqrt{X^2} = \sqrt{\frac{1}{T} \int\limits ^{t_0+T}_{t_0}{x^2(t)dt}}$&
			$k_{s}=\frac{X_{\mathrm{max}}}{X_{\mathrm{eff}}}$&
			&
			&
			\\
			\hline
			\includegraphics[width=2cm]{images/table_sine_wave.png} &
			$A\cdot\sin(t)$ &
			$\frac{2}{\pi} \approx 0.637$ &
			$\frac{\pi}{2\sqrt{2}} \approx 1.11$ &
			$\frac{1}{\sqrt{2}}\approx 0.707$ &
			$\sqrt{2}\approx 1.414$ &
			$0$ &
			$\frac{A^2}{2}$ &
			$\frac{A^2}{2}$ \\
			\hline	
			\includegraphics[width=2cm]{images/table_full-wave_rectified_sine.png} &
			$A\cdot|\sin(t)|$ &
			$\frac{2}{\pi} \approx 0.637$ &
			$\frac{\pi}{2\sqrt{2}} \approx 1.11$ &
			$\frac{1}{\sqrt{2}} \approx 0.707$ &
			$\sqrt{2} \approx 1.414$  &
			$\frac{2A}{\pi}$ & $\frac{A^2}{2}$ & $\frac{A^2}{2}-\frac{4A^2}{\pi^2}$
			\\
			\hline
			\includegraphics[width=2cm]{images/table_half-wave_rectified_sine.png} &
			$\begin{cases} A\cdot\sin (t) & 0<t<\pi  \\ 0 & \text{True}\end{cases}$ &
			$\frac{1}{\pi}\approx 0.318$ &
			$\frac{\pi}{2}\approx 1.571$ &
			$\frac{1}{2} = 0.5$	&
			2  &
			$\frac{A}{\pi}$ &
			$\frac{A^2}{4}$ & $\frac{A^2}{4}-\frac{A^2}{\pi^2}$
			\\
			\hline
			\includegraphics[width=2cm]{images/table_triangle_wave.png} &
			$A\cdot\Lambda(t)$ &
			$\frac{1}{2}= 0.5$ &
			$\frac{2}{\sqrt{3}}\approx 1.155$ &
			$\frac{1}{\sqrt{3}}
			\approx 0.557$ &
			$\sqrt{3} \approx 1.732$ &
			$0$ &
			$\frac{A^2}{3}$ &
			$\frac{A^2}{3}$ \\
			\hline	
			\includegraphics[width=2cm]{images/table_square_wave.png} &
			$\begin{cases} A & 0<x<t \\ 0 & \text{True}\end{cases}$ &
			$1$ &
			$1$ &
			$1$ &
			$1$ &
			$0$ &
			$A^2$ &
			$A^2$ \\
			\hline	
			DC&
			1&
			$1$ &
			$1$ &
			$1$ &
			$1$  &
			-&
			-&
			-\\
			\hline	
			\includegraphics[width=2cm]{images/table_pulse_wide_wave.png} &
			&
			$\frac{t_1}{T}$ & $\sqrt{\frac{T}{t_1}}$ & $\sqrt{\frac{t_1}{T}}$ & $\sqrt{\frac{T}{t_1}}$ &
			$A\frac{t}{T}$ &
			$A^2\frac{t}{T}$ &
			$\frac{A^2t}{T}-\frac{A^2t^2}{T^2}$\\
			\hline
		\end{tabular}
\end{sidewaystable}
\clearpage
\newpage
\section{Linux-Tipps}
\subsection{Allgemeine Tipps für Terminal}
\begin{table}[H]
	\begin{tabular}{|p{0.5\textwidth}|p{0.3\textwidth}|}
		\hline
		\textbf{Befehl} & \textbf{Code}\\
		\hline
		Verzeichnis foo & \lstinline|cd foo|\\
		\hline
		Verzeichnis retour & \lstinline|cd ..|\\
		\hline
		Files im Verzeichnis auflisten & \lstinline|ls|\\
		\hline
		Files im Verzeichnis auflisten mit Infos & \lstinline|ls -l|\\
		\hline
		Files im Verzeichnis auflisten auch versteckte& \lstinline|ls -a|\\
		\hline
		Auflisten mit Option a und l & \lstinline|ls -al|\\
		\hline
		Erstellt Verzeichnis & \lstinline|mkdir foo|\\
		\hline
		Löscht leeres Verzeichnis& \lstinline|rmdir foo|\\
		\hline
		Löscht Datei & \lstinline|rm name|\\
		\hline
		Gibt Anzahl Zeilen und Wörter im File aus& \lstinline|wc file|\\
		\hline
	\end{tabular}
\end{table}
\subsubsection{Kompilieren}
\begin{table}[H]
	\noindent
	\begin{tabular}{|p{0.15\textwidth}|p{0.6\textwidth}|}
		\hline
		\textbf{Befehl} & \textbf{Code}\\
		\hline
		\textbf{C} & \\
		Kompilieren & \lstinline|gcc program-source-code.c -o executable-file-name|\\
		& \lstinline|gcc test1.c -o test|\\
		& \lstinline|./test1|\\
		\hline
		\textbf{C++} & \\
	    Kompilieren & \lstinline|g++ program-source-code.cpp -o executable-file-name|\\
		& \lstinline|g++ test1.cpp -o test|\\
		& \lstinline|./test2|\\
		\hline 
		\textbf{Java} & \\
		Kompilieren &
		\lstinline|javac filename.java|\\
		Ausführen & 
		\lstinline|java filename|\\
		\hline 
	\end{tabular}
\end{table}

\clearpage
\pagebreak

\end{document}

%TODO Wichtige Sachen aus 4.Semester ergänzen wie Übertragungsfunktion