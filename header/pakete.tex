%TODO Packete ergänzen
%--------------------%
%File für alle Pakete

\usepackage{array} % Extending the array and tabular environment
\usepackage{textcomp} %Wird für Copyright-Symbol,Währungen, Musikalische-Symbole benötigt
\usepackage{graphicx} % Für Grafiken
\usepackage{tabularx} % Für Tabellen
\usepackage{amssymb} % Das mathtools package ist eine Erweiterung zum amsmath package.
\usepackage{mathtools} % Erweiterung amsmath mit Tools, Layoutsytle
\usepackage{mathabx} % Package mit vielen weiteren Mathe Symbolen
\usepackage{amsmath} % Für equitation
\usepackage{adjustbox} %adjustbox, minipage..
\usepackage{amsopn} %Für DeclareMathOperator
\usepackage{pxfonts} % Mathsymbols
\usepackage[automark]{scrpage2} % Header und Footer
\usepackage{multirow} % Create tabular cells spanning multiple rows
\usepackage{multicol} % In­ter­mix sin­gle and mul­ti­ple columns
\usepackage{rotating} % Rotation tools, including rotated fullpage floats
\usepackage{trfsigns} % Für Imaginär , Euler, Laplace
\usepackage{rotating} % Möglichkeit alles zu drehen
\usepackage{pdfpages} % Für include pdf-seiten
\usepackage{pbox}

%Deutsche Sprache mit Sonderzeichen und Umlauten
\usepackage[utf8]{inputenc}  %Zeichenart UTF-8
\usepackage[T1]{fontenc} %ä,ü...
\usepackage[ngerman]{babel}  %Silbentrennung und Rechtschreibung Deutsch

%Schriftart mit LuaLatex (alle Schriften aus Word möglich)
\usepackage{fontspec}
\setmainfont{Arial}

%Schriftart mit pdflatex Compiler
%\usepackage{helvet}
%\renewcommand{\familydefault}{\sfdefault}
%\fontfamily{phv}\selectfont

%Hyperlinks im Dokument
\usepackage[breaklinks,pdftex]{hyperref}

% Seitenränder für Formelsammlungen
\usepackage[left=1.27cm,right=1.27cm,top=0.80cm,bottom=0.80cm,includeheadfoot]{geometry}

