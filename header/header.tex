%BuG-Fix
%Package pdf Error: Driver file ................ not found
%If you have a luatex driver fail uncomment these lines
\RequirePackage{luatex85}
\def\pgfsysdriver{pgfsys-pdftex.def}
% Genereller Header
\documentclass[11pt,twoside,a4paper,fleqn]{article}
% Dateiencoding
%\usepackage{fontspec}
%\setmainfont{Calibri}
\usepackage[utf8]{inputenc}
\usepackage[T1]{fontenc}	%ä,ü...
% Seitenränder
\usepackage[left=1cm,right=1cm,top=0.5cm,bottom=0.5cm,includeheadfoot]{geometry}
% Sprachpaket
\usepackage[ngerman]{babel} % Silbentrennung und Rechtschreibung Englisch und Deutsch

%%%%%%%%%%%%%%%%%%%%%%%
%% Wichtige Packages %%
%%%%%%%%%%%%%%%%%%%%%%%
\usepackage{amsmath}                % Allgemeine Matheumgebungen									
\usepackage{amssymb}                % Fonts: msam,msbm, eufm & Mathesymbole, Mengen (lädt automatisch amsfonts)									
\usepackage{array}                  % \newcolumntype, \firsthline, ,\lasthline, m{width}, b{width}									
\usepackage{caption}                % Bildunterschriften									
\usepackage{enumitem}               % basic environments: enumerate, itemize, description									
\usepackage{fancybox}               % \fbox: \shad­ow­box, \dou­ble­box, \oval­box, \Oval­box									
\usepackage{fancyhdr}               % Seiten schöner gestalten, insbesondere Kopf- und Fußzeile									
\usepackage{floatflt}               % Textumflossene Abbildungen \begin{floatingfigure}[r]{Breite} : r rechts, l links, p links auf geraden Seiten und rechts auf ungeraden Seiten									
\usepackage{graphicx}               % \includegraphics[keyvals]{imagefile}, [draft]graphicx zeigt nur Namen und Rahmen an, [final] hebt diese option auf => Bild wird angezeigt    									
\usepackage{hyperref}               % Erstellt Verweise innerhalb und nach außerhalb eines PDF Dokumentes.									
\usepackage{lastpage}               % Bspw. : Page 1 of 3 => \thepage\ of \pageref{LastPage}									
\usepackage{listings}               % Erlaubt es Programmcode in der gewünschten Sprache zu hinterlegen (C++, Matlab,..). Definition der Sprache mit \lstset{language=name}..									
\usepackage{longtable}              % Longtable erlaubt es Tabellen zu erstellen die bei der nächsten Seite weiterlaufen. (Bricht automatisch um)									
\usepackage{mathabx}                % Mathesymbole									
\usepackage{mathrsfs}               % \mathscr (Benötigt für Fourierreihen-Symbol)									
%\usepackage{mathtools}              % Extension package to amsmath									
\usepackage{multicol}               % multicols-Umgebung \begin{multicols}{3} erzeugt Abschnitt mit 3 Spalten									
\usepackage{multirow}               % Tabelle: ermöglicht es Felder mehrerer Zeilen in einem zusammenzufassen									
\usepackage{pdflscape}              % adds PDF support to the environment 'landscape'									
\usepackage{pxfonts}                % Symbole, griechisches Alphabet, Integrale...									
\usepackage{rotating}               % sideways, turn{degree}, rotate{degree}, sidewaysfigure, sidewaystable Umgebung									
\usepackage{subcaption}             % Bildunterschriften für Subfigures									
\usepackage{tabularx}               % tabularx-Umgebung: Hat feste Gesamtbreite, \begin{tabularx}{\textwidth}{c c c c c} X: Spalte mit variabler Breite, l, c, r, p{breite}, m{breite}									
\usepackage{textcomp}               % text symbols: baht, bullet, copyright, musical-note, onequarter, section, yen									
\usepackage{tikz}                   % Tikz Umgebung zur Grafikerzeugung									
\usepackage{titlesec}               % Überschriften zu Textabstände
\usepackage{trfsigns}               % Transformationszeichen \laplace, \Laplace..									
\usepackage{trsym}                  % Weitere Laplace Zeichen erlaubt auch vertikale Transformationszeichen									
\usepackage{verbatim}               % verbatim, verbatim*, comment Umgebung									
\usepackage{wrapfig}                % Textumflossene Bilder und Tabellen, \begin{wrapfigure}[Zeilen]{Position}[Ueberhang]{Breite}									
\usepackage{xcolor}                 % \pagecolor{color}, \textcolor{color}{text}, \colorbox{color}{text}, \fcolorbox{border-color}{fill-color}{text}									
\usepackage{titlesec}
% Zum Bilder einfach in Tabellen einfügen (valign=t)
\usepackage[none]{hyphenat}
\usepackage[export]{adjustbox}
\usepackage{listings}
\usepackage{esint}
\usepackage{graphics}
\usepackage{pdfpages}
\usepackage{float}

%%%%%%%%%%%%%%%%%%%%
% Generelle Makros %
%%%%%%%%%%%%%%%%%%%%


\newcommand\tabbild[2][]{%
	\raisebox{0pt}[\dimexpr\totalheight+\dp\strutbox\relax][\dp\strutbox]{%
		\includegraphics[#1]{#2}%
	}%
}


%%%%%%%%%%
% Farben %
%%%%%%%%%%
\definecolor{black}{rgb}{0,0,0}
\definecolor{red}{rgb}{1,0,0}
\definecolor{white}{rgb}{1,1,1}
\definecolor{grey}{rgb}{0.8,0.8,0.8}
\definecolor{green}{rgb}{0,.8,0.05}
\definecolor{brown}{rgb}{0.603,0,0}
\definecolor{mymauve}{rgb}{0.58,0,0.82}


%%%%%%%%%%%%%%%%%%%%%%%%%%%%
% Mathematische Operatoren %
%%%%%%%%%%%%%%%%%%%%%%%%%%%%
\DeclareMathOperator{\sinc}{sinc}
\DeclareMathOperator{\sgn}{sgn}
\DeclareMathOperator{\Real}{Re}
\DeclareMathOperator{\Imag}{Im}
\DeclareMathOperator{\euler}{e}
\DeclareMathOperator{\cov}{cov}
\DeclareMathOperator{\PolyGrad}{PolyGrad}
\DeclareMathOperator{\gradient}{grad}
\DeclareMathOperator{\rotation}{rot}
\DeclareMathOperator{\divergenz}{div}
\DeclareMathOperator{\imaginär}{j}
\DeclareMathOperator\arccot{arccot}
\DeclareMathOperator\arcsec{arcsec}
\DeclareMathOperator\arcossec{arcossec}
\DeclareMathOperator\arsinh{arsinh}
\DeclareMathOperator\arcosh{arscosh}
\DeclareMathOperator\artanh{artanh}
\DeclareMathOperator\arcoth{arcoth}


%%%%%%%%%%%%%%%%%%%%%%%%%%%%
% Allgemeine Einstellungen %
%%%%%%%%%%%%%%%%%%%%%%%%%%%%
%Pdf Info
\hypersetup{pdfauthor={\authorname},pdftitle={\titleinfo},colorlinks=false}
\author{\authorname}
\title{\titleinfo}


%%%%%%%%%%%%%%%%%%%%%%%
% Kopf- und Fusszeile %
%%%%%%%%%%%%%%%%%%%%%%%
\pagestyle{fancy}
\fancyhf{}
%Linien oben und unten
\renewcommand{\headrulewidth}{0.5pt} 
\renewcommand{\footrulewidth}{0.5pt}

%Kopfzeile links bzw innen
\fancyhead[L]{\titleinfo{ }}
%Kopfzeile mitte
%\fancyhead[C]{}
%Kopfzeile rechts bzw. aussen
\fancyhead[R]{Seite \thepage { }von \pageref{LastPage}}

%Fusszeile links bzw. innen
\fancyfoot[L]{\footnotesize{\authorname}}
%Fusszeile mitte
\fancyfoot[C]{\footnotesize{Elektrotechnik@HSR}}
%Fusszeile rechts bzw. ausen
\fancyfoot[R]{\footnotesize{\today}}
% Einrücken verhindern versuchen
\setlength{\parindent}{0pt}
