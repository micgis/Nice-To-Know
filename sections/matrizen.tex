\section{Matrizen}
\subsection{Gaussverfahren}
Durch Addition und Subtraktion einzelner Zeilen (auch von Vielfachen einer
Zeile) werden einzelne Stellen auf Null gebracht.\\
\vspace{1.0pt}
$\begin{bmatrix}
a_{11} & a_{12}& \ldots & a_{1n}\\
a_{21}& &\ldots & \\
\ldots \\
a_{n1} & & \ldots & a_{nn}    			
\end{bmatrix}=
\begin{bmatrix}
a_{11} & a_{12}& \ldots & a_{1n}\\
k a_{21}-n a_{11}& ka_{22}-n a_{12}&\ldots & k a_{2n} - n a_{1n}\\
\ldots \\
a_{n1} & & \ldots & a_{nn}    			
\end{bmatrix}$ \\
\vspace{1.0pt}
Die n * erste Zeile wurde von der k * zweiten Zeile abgezogen ($a_{2.}= 
k a_{2.}- n a_{1.}$)

\subsection{Determinante}
\begin{minipage}[t]{6cm}
	\textbf{2x2 Matrix}    
	\[ \det \begin{bmatrix}
	a & b \\
	c & d
	\end{bmatrix} = ad - bc \]
\end{minipage}
\begin{minipage}[t]{12cm}
	\textbf{3x3 Matrix}
	\[\begin{bmatrix}
	a & b & c \\
	d & e & f \\
	g & h & i 
	\end{bmatrix} = aei + bfg + cdh - ceg - afh - bdi \]
\end{minipage}

\subsubsection{Grössere Matrizen}
$$A\epsilon M_n: \det A =    
\begin{vmatrix}
a_{11} & a_{12}& \ldots & a_{1n}\\
a_{21}& &\ldots & \\
\ldots \\
a_{n1} & & \ldots & a_{nn}    			
\end{vmatrix}=
(-1)^{1+1}a_{11}D_{11} + (-1)^{1+2}a_{12}D_{12}+ \ldots +
(-1)^{1+n}a_{1n}D_{1n}$$
\\
\textbf{Unterdeterminante}
$$D_{11}=
\begin{vmatrix}
a_{22} & \ldots & a_{2n}\\
\ldots\\
a_{n2}& \ldots & a_{nn}
\end{vmatrix} 	\\
\qquad
D_{12}=
\begin{vmatrix}
a_{21} & a_{23}& \ldots & a_{2n}\\
\ldots\\
a_{n1}& a_{n3}&\ldots & a_{nn}
\end{vmatrix}$$\\
$D_{ij}$ die (n-1)$ \times $(n-1)-Untermatrix von D ist, die durch Streichen der
i-ten Zeile und j-ten Spalte entsteht.\\
Diese Methode ist zu empfehlen, wenn die Matrix in einer Zeile oder Spalte
bis auf eine Stelle nur Nullen aufweisst.
Dies lässt sich meist mit dem Gausverfahren bewerkstelligen.

\subsection{Inverse Matrix}
Existiert nur wenn Matrix regulär: $\det A \neq 0$\\
\begin{minipage}{7cm}
	\textbf{2x2 Matrix:}    
	$$ A^{-1} = \begin{bmatrix} a & b \\ c & d \\ \end{bmatrix}^{-1} = \frac{1}{ad
		- bc} \begin{bmatrix} d & -b \\ -c & a \\ \end{bmatrix} $$
\end{minipage}
\begin{minipage}{11cm}
	\textbf{3x3 Matrix:}
	$$  A^{-1} = \begin{bmatrix} a & b & c\\ d & e & f \\ g & h & i \\ \end{bmatrix}^{-1} =
	\frac{1}{\det(A)} \begin{bmatrix} ei - fh & ch - bi & bf - ce \\ fg - di & ai
	- cg & cd - af \\ dh - eg & bg - ah & ae - bd \end{bmatrix} $$
\end{minipage}\\

\begin{multicols}{2}
	\subsection{Transponierte Matrix}
	Transponierte Matrix:  $A^T=[a_{ik}^T]=[a_{ki}]$ \\
	vertauschen der Zeilen mit Spalten
	
	\subsection{Einheitsmatrix}
	Einheitsmatrix: $I=E= 
	\begin{bmatrix} 
	1&0 & 0\\
	0&1&0\\
	0&0&1                               
	\end{bmatrix}$	
\end{multicols}

\subsubsection{Grössere Matrizen}
Alle Elemete elementweise invertieren - Kehrwert. $\quad \Rightarrow \quad $\textit{Gilt nur wenn alle Elemente auf der Hauptdiagonale $\neq 0$ sind.}\\

$A^{-1}= \begin{bmatrix}
a_{11} & a_{12}& \ldots & a_{1n}\\
a_{21}& &\ldots & \\
\ldots \\
a_{n1} & & \ldots & a_{nn}    			
\end{bmatrix}^{-1}$

\begin{enumerate}
	\item $A^T$ bestimmen (Zeilen und Spalten vertauschen)
	$ \; A^{T}= \begin{bmatrix}
	a_{11} & a_{21}& \ldots & a_{n1}\\
	a_{12}& &\ldots & \\
	\ldots \\
	a_{1n} & & \ldots & a_{nn}    			
	\end{bmatrix}$	
	\item Bei $A^T$ jedes Element durch Unterdeterminante ersetzen
	$\;A^*=	\begin{bmatrix}
	(-1)^{1+1}D_{11} &  \ldots	& (-1)^{1+n} D_{1n}\\
	\ldots\\
	(-1)^{n+1} D_{n1}& \ldots  & (-1)^{n+n} D_{nn}
	\end{bmatrix}$
	\item $A^{-1} = \frac{A^*}{\det A}$ 
\end{enumerate}


	\subsection{Diagonalisierung}
	\begin{enumerate}
		\item Eigenwerte $\lambda$ ausrechnen: $\det (A - I_n \lambda)=0$
		\item Eigenvektoren $\vec{v}$ bilden: $(A- \lambda I_n)\vec{v}=0$
		\item Transformationsmatrix: $T= [\vec{v_1} \ldots \vec{v_n}]$
		\item $T^{-1}$ berechnen (Achtung ist A symmetrisch, dh. $A^T=A$ und
		oder alle EV senktrecht zueinander, dann $T^{-1}=T^T$)
		\item $D=\begin{bmatrix}
		\lambda_1 &0 &0\\
		0& \lambda_2 &0\\
		0& 0& \lambda_3
		\end{bmatrix} = A_{diag} = T^{-1}AT$
	\end{enumerate}
	
	\subsection{Eigenwerte}
	Die Eigenwerte $\lambda$ erhält man folgendermassen ($I$ ist die Einheitsmatrix):
	\[ |\lambda I - A| = 0 \qquad \text{nach } \lambda \text{ auflösen} \]
