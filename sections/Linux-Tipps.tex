\section{Linux-Tipps}
\subsection{Allgemeine Tipps für Terminal}
\begin{table}[H]
	\begin{tabular}{|p{0.5\textwidth}|p{0.3\textwidth}|}
		\hline
		\textbf{Befehl} & \textbf{Code}\\
		\hline
		Verzeichnis foo & \lstinline|cd foo|\\
		\hline
		Verzeichnis retour & \lstinline|cd ..|\\
		\hline
		Files im Verzeichnis auflisten & \lstinline|ls|\\
		\hline
		Files im Verzeichnis auflisten mit Infos & \lstinline|ls -l|\\
		\hline
		Files im Verzeichnis auflisten auch versteckte& \lstinline|ls -a|\\
		\hline
		Auflisten mit Option a und l & \lstinline|ls -al|\\
		\hline
		Erstellt Verzeichnis & \lstinline|mkdir foo|\\
		\hline
		Löscht leeres Verzeichnis& \lstinline|rmdir foo|\\
		\hline
		Löscht Datei & \lstinline|rm name|\\
		\hline
		Gibt Anzahl Zeilen und Wörter im File aus& \lstinline|wc file|\\
		\hline
	\end{tabular}
\end{table}
\subsubsection{Kompilieren}
\begin{table}[H]
	\noindent
	\begin{tabular}{|p{0.15\textwidth}|p{0.6\textwidth}|}
		\hline
		\textbf{Befehl} & \textbf{Code}\\
		\hline
		\textbf{C} & \\
		Kompilieren & \lstinline|gcc program-source-code.c -o executable-file-name|\\
		& \lstinline|gcc test1.c -o test|\\
		& \lstinline|./test1|\\
		\hline
		\textbf{C++} & \\
	    Kompilieren & \lstinline|g++ program-source-code.cpp -o executable-file-name|\\
		& \lstinline|g++ test1.cpp -o test|\\
		& \lstinline|./test2|\\
		\hline 
		\textbf{Java} & \\
		Kompilieren &
		\lstinline|javac filename.java|\\
		Ausführen & 
		\lstinline|java filename|\\
		\hline 
	\end{tabular}
\end{table}

\clearpage
\pagebreak